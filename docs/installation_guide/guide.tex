%% *****************************************************************************
%% 
%% Alfresco PDF Sign - Installation Guide
%% 
%% Description: 
%% This document provides a comprehensive guide for the installation, 
%% configuration, and verification of the Alfresco PDF Sign plugin. 
%% The plugin enables PDF signing capabilities within Alfresco Community 
%% Edition 7.4.2. Detailed instructions are provided to ensure successful 
%% deployment and integration with your existing Alfresco environment.
%% 
%% Author: 
%% Rober de Avila Abraira
%% 
%% Version: 
%% 1.0.0
%% 
%% Date: 
%% 2024/08/20
%% 
%% License: 
%% This document is licensed under the Apache License, Version 2.0. 
%% You may not use this file except in compliance with the License. 
%% A copy of the License can be obtained at:
%% http://www.apache.org/licenses/LICENSE-2.0
%% Unless required by applicable law or agreed to in writing, this 
%% document is distributed on an "AS IS" BASIS, WITHOUT WARRANTIES 
%% OR CONDITIONS OF ANY KIND, either express or implied. See the 
%% License for the specific language governing permissions and 
%% limitations under the License.
%% 
%% *****************************************************************************


\documentclass{ol-softwaremanual}

% Packages used
\usepackage[utf8]{inputenc} % for using Spanish characters
\usepackage{graphicx}  % for including images
\usepackage{microtype} % for typographical enhancements and better justification
\usepackage{ragged2e} % for justifying content
\usepackage{hyperref}  % for hyperlinks
\usepackage{listings} % For code listings
\usepackage{xcolor} % For customizing colors
\usepackage[a4paper,top=4.2cm,bottom=4.2cm,left=3.5cm,right=3.5cm]{geometry} % for setting page size and margins

\renewcommand{\contentsname}{Contenido}

% Custom macros used in this example document
\newcommand{\doclink}[2]{\href{#1}{#2}\footnote{\url{#1}}}
\newcommand{\cs}[1]{\texttt{\textbackslash #1}}

% Frontmatter data; appears on title page
\title{Guía de Instalación \\ Alfresco PDF Sign}
\version{1.0.0}
\author{Rober de Avila Abraira}
\softwarelogo{\includegraphics[width=8cm]{logo}}


\begin{document}

\maketitle

\tableofcontents
\newpage


\section{Introducción}

El plugin \textbf{Alfresco PDF Sign} ha sido desarrollado para permitir la firma digital de documentos PDF dentro del entorno de \textbf{Alfresco Community Edition 7.4.2}. Este plugin ofrece una integración directa con el sistema de gestión documental de Alfresco, facilitando la validación y autenticación de documentos mediante firmas digitales, lo que es crucial para garantizar la integridad y autenticidad en flujos de trabajo que manejan información sensible o regulada.

Esta guía de instalación proporciona instrucciones detalladas y paso a paso para la configuración, instalación y verificación del plugin. Está dirigida a administradores de sistemas y desarrolladores que trabajan con Alfresco y necesitan implementar una solución robusta para la firma de documentos dentro de su infraestructura existente.

\subsection{Objetivos de la Guía}
Al completar esta guía, serás capaz de:

\begin{itemize}
	\item Instalar y configurar el plugin Alfresco PDF Sign en un entorno Alfresco existente.
	\item Verificar que la instalación se haya realizado correctamente.
	\item Resolver posibles problemas durante el proceso de instalación.
	\item Configurar ajustes adicionales necesarios para adaptarse a las necesidades específicas de tu organización.
\end{itemize}

\subsection{Requisitos Previos}
Antes de proceder con la instalación, asegúrate de cumplir con los siguientes requisitos:

\begin{itemize}
	\item \textbf{Alfresco Community Edition 7.4.2} debe estar instalado y en funcionamiento.
	\item Debes tener acceso administrativo al servidor donde está instalado Alfresco.
	\item Conocimientos básicos de administración de Alfresco y del sistema operativo subyacente.
\end{itemize}

\subsubsection{Opcionales:}
\begin{itemize}
	\item \textbf{Maven:} Solo es necesario si planeas construir la librería del plugin desde su código fuente. Si ya tienes el archivo .amp o .jar del plugin y no necesitas realizar ninguna compilación, Maven no es necesario.
	\item \textbf{Docker y Docker Compose:} Opcionales para gestionar la instalación y el despliegue del plugin en contenedores. Utiliza esta opción si prefieres una instalación basada en contenedores en lugar de una instalación directa en Tomcat.
\end{itemize}

\subsection{Opciones de Instalación}
El plugin \textbf{Alfresco PDF Sign} puede instalarse mediante diferentes métodos, dependiendo de las necesidades y la infraestructura disponible en tu entorno:

\begin{enumerate}
	\item \textbf{Instalación usando Docker (Recomendado)} \\
	Esta opción facilita la gestión y el despliegue de Alfresco y sus componentes en contenedores, lo que puede simplificar la administración del entorno. Es ideal para aquellos que prefieren una solución rápida y reproducible utilizando tecnología de contenedores.
	
	\item \textbf{Instalación directa en Tomcat} \\
	El plugin puede instalarse directamente en una instancia de Tomcat que ya esté ejecutando Alfresco. Este método es adecuado si ya tienes un entorno Alfresco configurado y prefieres no utilizar contenedores.
\end{enumerate}


\subsection{Opciones de Compilación}
Al instalar el plugin Alfresco PDF Sign, tienes dos opciones principales según cómo prefieras manejar la compilación:

\begin{itemize}
	\item \textbf{Compilar el archivo .amp:} Si deseas personalizar el plugin o necesitas construirlo tú mismo, puedes generar el archivo .amp utilizando herramientas de construcción como Maven. Esta opción es ideal para desarrolladores o aquellos que requieren un mayor control sobre el proceso de instalación.
	\item \textbf{Utilizar un archivo .amp precompilado (Recomendado):} Si prefieres una instalación más sencilla y directa, puedes optar por un archivo .amp ya compilado. Esta opción te permite instalar el plugin sin necesidad de realizar la compilación, lo que simplifica y acelera el proceso.
\end{itemize}

\section{Instalación del Plugin Alfresco PDF Sign}
\subsection{Preparación del Entorno}
\subsubsection{Verificación de los Requisitos Previos}
Antes de iniciar la instalación, asegúrate de que todos los requisitos previos mencionados en la sección 1.2 estén cumplidos. Esto incluye la confirmación de que \textbf{Alfresco Community Edition 7.4.2} está en funcionamiento y que tienes acceso administrativo al servidor.

\subsubsection{Descarga del Plugin}
Puedes obtener el plugin de dos maneras, dependiendo de si deseas construir el archivo .amp desde el código fuente o utilizar una versión precompilada:

\begin{itemize}
	\item \textbf{Código Fuente:} Si deseas personalizar el plugin o necesitas construirlo, clona el repositorio desde GitHub:
\begin{lstlisting}[
	language=bash,
	frame=single,
	basicstyle=\ttfamily\scriptsize\color{white}, % Set text to white and slightly larger
	backgroundcolor=\color{black}, % Set background to black
	frame=single, % Frame around the code block
	framerule=0pt, % Remove the frame rule (optional)
]
  git clone https://github.com/abraira85/alfresco-pdf-sign.git
  cd alfresco-pdf-sign
\end{lstlisting}
	\item \textbf{Versión Precompilada:} Si prefieres una instalación más rápida y no necesitas modificar el plugin, puedes descargar el archivo .amp precompilado desde la sección de Releases del mismo repositorio en GitHub.
\end{itemize}

\end{document}

